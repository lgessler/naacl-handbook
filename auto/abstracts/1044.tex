A central component of rational behavior is logical inference: the process of determining which conclusions follow from a set of premises. Psychologists have documented several ways in which humans' inferences deviate from the rules of logic. Do language models, which are trained on text generated by humans, replicate such human biases, or are they able to overcome them? Focusing on the case of syllogisms---inferences from two simple premises---we show that, within the PaLM\textasciitilde{}2 family of transformer language models, larger models are more logical than smaller ones, and also more logical than humans. At the same time, even the largest models make systematic errors, some of which mirror human reasoning biases: they show sensitivity to the (irrelevant) ordering of the variables in the syllogism, and draw confident but incorrect inferences from particular syllogisms (syllogistic fallacies). Overall, we find that language models often mimic the human biases included in their training data, but are able to overcome them in some cases.