While major languages often enjoy substantial attention and resources, the linguistic diversity across the globe encompasses a multitude of smaller, indigenous, and regional languages that lack the same level of computational support. One such region is the Caribbean. While commonly labeled as "English speaking", the ex-British Caribbean region consists of a myriad of Creole languages thriving alongside English. In this paper, we present Guylingo: a comprehensive corpus designed for advancing NLP research in the domain of Creolese (Guyanese English-lexicon Creole), the most widely spoken language in the culturally rich nation of Guyana. We first outline our framework for gathering and digitizing this diverse corpus, inclusive of colloquial expressions, idioms, and regional variations in a low-resource language. We then demonstrate the challenges of training and evaluating NLP models for machine translation for Creolese. Lastly, we discuss the unique opportunities presented by recent NLP advancements for accelerating the formal adoption of Creole languages as official languages in the Caribbean.