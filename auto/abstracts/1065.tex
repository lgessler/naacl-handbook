Text-based false information permeates online discourses, yet evidence of people's ability to discern truth from such deceptive textual content is scarce. We analyze a novel TV game show data where conversations in a high-stake environment between individuals with conflicting objectives result in lies. We investigate the manifestation of potentially verifiable language cues of deception in the presence of objective truth, a distinguishing feature absent in previous text-based deception datasets. We show that there exists a class of detectors (algorithms) that have similar truth detection performance compared to human subjects, even when the former accesses only the language cues while the latter engages in conversations with complete access to all potential sources of cues (language and audio-visual). Our model, built on a large language model, employs a bottleneck framework to learn discernible cues to determine truth, an act of reasoning in which human subjects often perform poorly, even with incentives. Our model detects novel but accurate language cues in many cases where humans failed to detect deception, opening up the possibility of humans collaborating with algorithms and ameliorating their ability to detect the truth.