Fairness-related assumptions about what constitute appropriate NLG system behaviors range from invariance, where systems are expected to behave identically for social groups, to adaptation, where behaviors should instead vary across them. To illuminate tensions around invariance and adaptation, we conduct five case studies, in which we perturb different types of identity-related language features (names, roles, locations, dialect, and style) in NLG system inputs. Through these cases studies, we examine people's expectations of system behaviors, and surface potential caveats of these contrasting yet commonly held assumptions. We find that motivations for adaptation include social norms, cultural differences, feature-specific information, and accommodation; in contrast, motivations for invariance include perspectives that favor prescriptivism, view adaptation as unnecessary or too difficult for NLG systems to do appropriately, and are wary of false assumptions. Our findings highlight open challenges around what constitute "fair" or "good" NLG system behaviors.