Large language models (LLMs) are dramatically influencing AI research, spurring discussions on what has changed so far and how to shape the field's future. To clarify such questions, we analyze a new dataset of 16,979 LLM-related arXiv papers, focusing on recent trends in 2023 vs. 2018-2022. First, we study disciplinary shifts: LLM research increasingly considers societal impacts, evidenced by 20$\times$ growth in LLM submissions to the Computers and Society sub-arXiv. An influx of new authors -- half of all first authors in 2023 -- are entering from non-NLP fields of CS, driving disciplinary expansion. Second, we study industry and academic publishing trends. Surprisingly, industry accounts for a smaller publication share in 2023, largely due to reduced output from Google and other Big Tech companies; universities in Asia are publishing more. Third, we study institutional collaboration: while industry-academic collaborations are common, they tend to focus on the same topics that industry focuses on rather than bridging differences. The most prolific institutions are all US- or China-based, but there is very little cross-country collaboration. We discuss implications around (1) how to support the influx of new authors, (2) how industry trends may affect academics, and (3) possible effects of (the lack of) collaboration.