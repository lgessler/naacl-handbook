We present COPAL-ID, a novel, public Indonesian language common sense reasoning dataset. Unlike the previous Indonesian COPA dataset (XCOPA-ID), COPAL-ID incorporates Indonesian local and cultural nuances, and therefore, provides a more natural portrayal of day-to-day causal reasoning within the Indonesian cultural sphere. Professionally written by natives from scratch, COPAL-ID is more fluent and free from awkward phrases, unlike the translated XCOPA-ID. In addition, we present COPALID in both standard Indonesian and in Jakartan Indonesian–a dialect commonly used in daily conversation. COPAL-ID poses a greater challenge for existing open-sourced and closed state-of-the-art multilingual language models, yet is trivially easy for humans. Our findings suggest that general multilingual models struggle to perform well, achieving 66.91\% accuracy on COPAL-ID. South-East Asian-specific models achieve slightly better performance of 73.88\% accuracy. Yet, this number still falls short of near-perfect human performance. This shows that these language models are still way behind in comprehending the local nuances of Indonesian.