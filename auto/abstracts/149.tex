Our research integrates graph data with Large Language Models (LLMs), which, despite their advancements in various fields using large text corpora, face limitations in encoding entire graphs due to context size constraints. This paper introduces a new approach to encoding a graph with diverse modalities, such as text, image, and motif, coupled with prompts to approximate a graph's global connectivity, thereby enhancing LLMs' efficiency in processing complex graph structures. The study also presents GraphTMI, a novel benchmark for evaluating LLMs in graph structure analysis, focusing on homophily, motif presence, and graph difficulty. Key findings indicate that the image modality, especially with vision-language models like GPT-4V, is superior to text in balancing token limits and preserving essential information and comes close to prior graph neural net (GNN) encoders. Furthermore, the research assesses how various factors affect the performance of each encoding modality and outlines the existing challenges and potential future developments for LLMs in graph understanding and reasoning tasks. Our code and data are publicly available on our project page - https://minnesotanlp.github.io/GraphLLM/