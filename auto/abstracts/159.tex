The permanence of online content combined with the enhanced authorship identification techniques calls for stronger computational methods to protect the identity and privacy of online authorship when needed, e.g., blind reviews for scientific papers, anonymous online reviews, or anonymous interactions in the mental health forums. In this paper, we propose an unsupervised inference-time approach to authorship obfuscation to address the unique challenges of authorship obfuscation: lack of supervision data for diverse authorship and domains, and the need for a sufficient level of revision beyond simple paraphrasing to obfuscate the authorship, all the while preserving the original content and fluency. We introduce JAMDEC, a user-controlled, inference-time algorithm for authorship obfuscation that can be in principle applied to any text and authorship. Our approach builds on small language models such as GPT2-XL in order to help avoid disclosing the original content to proprietary LLM’s APIs, while also reducing the performance gap between small and large language models via algorithmic enhancement. The key idea behind our approach is to boost the creative power of smaller language models through constrained decoding, while also allowing for user-specified controls and flexibility. Experimental results demonstrate that our approach based on GPT2-XL outperforms previous state-of-the-art methods based on comparably small models, while performing competitively against GPT3.5 175B, a propriety model that is two orders of magnitudes larger.