Structured data, prevalent in tables, databases, and knowledge graphs, poses a significant challenge in its representation. With the advent of large language models (LLMs), there has been a shift towards linearization-based methods, which process structured data as sequential token streams, diverging from approaches that explicitly model structure, often as a graph. Crucially, there remains a gap in our understanding of how these linearization-based methods handle structured data, which is inherently non-linear. This work investigates the linear handling of structured data in encoder-decoder language models, specifically T5. Our findings reveal the model's ability to mimic human-designed processes such as schema linking and syntax prediction, indicating a deep, meaningful learning of structure beyond simple token sequencing. We also uncover insights into the model's internal mechanisms, including the ego-centric nature of structure node encodings and the potential for model compression due to modality fusion redundancy. Overall, this work sheds light on the inner workings of linearization-based methods and could potentially provide guidance for future research.