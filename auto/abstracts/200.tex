Queer youth face increased mental health risks, such as depression, anxiety, and suicidal ideation. Hindered by negative stigma, they often avoid seeking help and rely on online resources, which may provide incompatible information. Although access to a supportive environment and reliable information is invaluable, many queer youth worldwide have no access to such support. However, this could soon change due to the rapid adoption of Large Language Models (LLMs) such as ChatGPT. This paper aims to comprehensively explore the potential of LLMs to revolutionize emotional support for queers.  To this end, we conduct a qualitative and quantitative analysis of LLM's interactions with queer-related content. To evaluate response quality, we develop a novel ten-question scale that is inspired by psychological standards and expert input. We apply this scale to score several LLMs and human comments to posts where queer youth seek advice and share experiences.  We find that LLM responses are supportive and inclusive, outscoring humans. However, they tend to be generic, not empathetic enough, and lack personalization, resulting in nonreliable and potentially harmful advice.  We discuss these challenges, demonstrate that a dedicated prompt can improve the performance, and propose a blueprint of an LLM-supporter that actively (but sensitively) seeks user context to provide personalized, empathetic, and reliable responses.  Our annotated dataset is available for further research. *https://github.com/nitaytech/LGBTeenDataset