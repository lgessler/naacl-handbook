Scaling high-quality tutoring remains a major challenge in education. Due to growing demand, many platforms employ novice tutors who, unlike experienced educators, struggle to address student mistakes and thus fail to seize prime learning opportunities. Our work explores the potential of large language models (LLMs) to close the novice-expert knowledge gap in remediating math mistakes. We contribute Bridge, a method that uses cognitive task analysis to translate an expert's latent thought process into a decision-making model for remediation. This involves an expert identifying (A) the student's error, (B) a remediation strategy, and (C) their intention before generating a response. We construct a dataset of 700 real tutoring conversations, annotated by experts with their decisions. We evaluate state-of-the-art LLMs on our dataset and find that the expert's decision-making model is critical for LLMs to close the gap: responses from GPT4 with expert decisions (e.g., ``simplify the problem'') are +76\% more preferred than without. Additionally, context-sensitive decisions are critical to closing pedagogical gaps: random decisions decrease GPT4's response quality by -97\% than expert decisions. Our work shows the potential of embedding expert thought processes in LLM generations to enhance their capability to bridge novice-expert knowledge gaps. Our dataset and code can be found at: https://github.com/rosewang2008/bridge.