The NLP community typically relies on performance of a model on a held-out test set to assess  generalization. Performance drops observed in datasets outside of official test sets are generally  attributed to "out-of-distribution" effects.  Here, we explore the foundations of generalizability and study the  factors that affect  it,  articulating  lessons from clinical studies. In clinical research,  generalizability is an act of reasoning that depends on  (a) *internal validity* of experiments to ensure controlled measurement of cause and effect, and (b) *external validity* or transportability of the results to the wider population. We demonstrate how learning spurious correlations, such as the distance between entities in  relation extraction tasks, can affect a model's internal validity and in turn adversely impact  generalization.  We, therefore, present the need to ensure internal validity when building machine learning models in  NLP. Our recommendations also apply to generative large language models, as they are  known to be sensitive   to even minor semantic preserving alterations. We also propose adapting the idea of *matching* in randomized controlled trials and observational studies to  NLP evaluation to measure causation.