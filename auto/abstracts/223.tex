Many NLP researchers are experiencing an existential crisis triggered by the astonishing success of ChatGPT and other systems based on large language models (LLMs). After such a disruptive change to our understanding of the field, what is left to do? Taking a historical lens, we look for guidance from the first era of LLMs, which began in 2005 with large $n$-gram models for machine translation (MT). We identify durable lessons from the first era, and more importantly, we identify evergreen problems where NLP researchers can continue to make meaningful contributions in areas where LLMs are ascendant. We argue that disparities in scale are transient and researchers can work to reduce them; that data, rather than hardware, is still a bottleneck for many applications; that meaningful realistic evaluation is still an open problem; and that there is still room for speculative approaches.