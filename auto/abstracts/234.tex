We introduce a new problem KTRL+F, a knowledge-augmented in-document search that necessitates real-time identification of all semantic targets within a document with the awareness of external sources through a single natural query. KTRL+F addresses following unique challenges for in-document search: 1) utilizing knowledge outside the document for extended use of additional information about targets, and 2) balancing between real-time applicability with the performance. We analyze various baselines in KTRL+F and find limitations of existing models, such as hallucinations, high latency, or difficulties in leveraging external knowledge. Therefore, we propose a Knowledge-Augmented Phrase Retrieval model that shows a promising balance between speed and performance by simply augmenting external knowledge in phrase embedding. We also conduct a user study to verify whether solving KTRL+F can enhance search experience for users. It demonstrates that even with our simple model, users can reduce the time for searching with less queries and reduced extra visits to other sources for collecting evidence. We encourage the research community to work on KTRL+F to enhance more efficient in-document information access.