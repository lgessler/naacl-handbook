In text ranking, it is generally believed that the cross-encoders already gather sufficient token interaction information via the attention mechanism in the hidden layers. However, our results show that the cross-encoders can consistently benefit from additional token interaction in the similarity computation at the last layer. We introduce CELI (Cross-Encoder with Late Interaction), which incorporates a late interaction layer into the current cross-encoder models. This simple method brings 5\% improvement on BEIR without compromising in-domain effectiveness or search latency. Extensive experiments show that this finding is consistent across different sizes of the cross-encoder models and the first-stage retrievers. Our findings suggest that boiling all information into the [CLS] token is a suboptimal use for cross-encoders, and advocate further studies to investigate its relevance score mechanism.