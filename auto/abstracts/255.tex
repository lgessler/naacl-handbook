Large language models (LLMs) have demonstrated considerable success in various natural language processing tasks, but open-source LLMs have yet to attain state-of-the-art performance in Neural Machine Translation (NMT). Nevertheless, their significant performance in tasks demanding a broad understanding and contextual processing shows their potential for translation. To exploit these abilities, we investigate using LLMs for MT and explore recent parameter-efficient fine-tuning techniques. Surprisingly, our initial experiments found that fine-tuning with Q-LoRA for translation purposes led to performance improvements in terms of BLEU but degradation in COMET compared to in-context learning. To overcome this, we propose an alternative approach: adapting LLMs as Automatic Post-Editors (APE) rather than direct translators. Building on the ability of the LLM to handle long sequences, we also propose extending our approach to document-level translation. We show that leveraging Low-Rank-Adapter fine-tuning for APE can yield significant improvements across both sentence and document-level metrics while generalizing to out-of-domain data. Most notably, we achieve a state-of-the-art accuracy rate of 88.7\textbackslash\{\}\% on the ContraPro test set, which assesses the model's ability to resolve pronoun ambiguities when translating from English to German. Lastly, during manual post-editing for document-level translation, the source sentences are iteratively annotated, which can be used to refine further translations in the document. Here, we demonstrate that leveraging human corrections can significantly reduce the number of edits required for subsequent translations.