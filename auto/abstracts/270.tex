The field of relation extraction (RE) is experiencing a notable shift towards generative relation extraction (GRE), leveraging the capabilities of large language models (LLMs). However, we discovered that traditional relation extraction (RE) metrics like precision and recall fall short in evaluating GRE methods. This shortfall arises because these metrics rely on exact matching with human-annotated reference relations, while GRE methods often produce diverse and semantically accurate relations that differ from the references. To fill this gap, we introduce GenRES for a multi-dimensional assessment in terms of the topic similarity, uniqueness, granularity, factualness, and completeness of the GRE results. With GenRES, we empirically identified that (1) precision/recall fails to justify the performance of GRE methods; (2) human-annotated referential relations can be incomplete; (3) prompting LLMs with a fixed set of relations or entities can cause hallucinations. Next, we conducted a human evaluation of GRE methods that shows GenRES is consistent with human preferences for RE quality. Last, we made a comprehensive evaluation of fourteen leading LLMs using GenRES across document, bag, and sentence level RE datasets, respectively, to set the benchmark for future research in GRE