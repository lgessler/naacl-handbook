It is well-known that speakers who entrain to one another have more successful conversations than those who do not. Previous research has shown that interlocutors entrain on linguistic features in both written and spoken $\emph{monolingual}$ domains. More recent work on $\emph{code-switched}$ communication has also shown preliminary evidence of entrainment on certain aspects of code-switching (CSW). However, such studies of entrainment in code-switched domains have been extremely few and restricted to human-machine textual interactions. Our work studies code-switched spontaneous speech between humans, finding that (1) patterns of written and spoken entrainment in monolingual settings largely generalize to code-switched settings, and (2) some patterns of entrainment on code-switching in dialogue agent-generated text generalize to spontaneous code-switched speech. Our findings give rise to important implications for the potentially "universal" nature of entrainment as a communication phenomenon, and potential applications in inclusive and interactive speech technology.