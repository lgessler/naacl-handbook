We present a game-theoretic model of pragmatics that we call ReCo (for Regularized Conventions). This model formulates pragmatic communication as a game in which players are rewarded for communicating successfully and penalized for deviating from a shared, “default” semantics. As a result, players assign utterances context-dependent meanings that jointly optimize communicative success and naturalness with respect to speakers’ and listeners’ background knowledge of language. By using established game-theoretic tools to compute equilibrium strategies for this game, we obtain principled pragmatic language generation procedures with formal guarantees of communicative success. Across several datasets capturing real and idealized human judgments about pragmatic implicature, ReCo matches, or slightly improves upon, predictions made by Iterated Best Response and Rational Speech Acts models of language understanding.