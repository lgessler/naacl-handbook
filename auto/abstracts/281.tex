Reference-based metrics that operate at the sentence-level typically outperform quality estimation metrics, which have access only to the source and system output. This is unsurprising, since references resolve ambiguities that may be present in the source. In this paper, we investigate whether additional source context can effectively substitute for a reference. We present a metric named SLIDE (SLIding Document Evaluator), which operates on blocks of sentences. SLIDE leverages a moving window that slides over each document in the test set, feeding each chunk of sentences into an unmodified, off-the-shelf quality estimation model. We find that SLIDE obtains significantly higher pairwise system accuracy than its sentence-level baseline, in some cases even eliminating the gap with reference-base metrics. This suggests that source context may provide the same information as a human reference in disambiguating source ambiguities. This finding is especially pertinent for reference-free document-level evaluation, wherein SLIDE could provide higher-quality pairwise system assessments while only requiring document boundary annotations.