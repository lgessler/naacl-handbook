Eliciting feedback from end users of NLP models can be beneficial for improving models. However, how should we present model responses to users so they are most amenable to be corrected from user feedback? Further, what properties do users value to understand and trust responses? We answer these questions by analyzing the effect of rationales (or explanations) generated by QA models to support their answers. We specifically consider decomposed QA models that first extract an intermediate rationale based on a context and a question and then use solely this rationale to answer the question. A rationale outlines the approach followed by the model to answer the question. Our work considers various formats of these rationales that vary according to well-defined properties of interest. We sample rationales from language models using few-shot prompting for two datasets, and then perform two user studies. First, we present users with incorrect answers and corresponding rationales in various formats and ask them to provide natural language feedback to revise the rationale. We then measure the effectiveness of this feedback in patching these rationales through in-context learning. The second study evaluates how well different rationale formats enable users to understand and trust model answers, when they are correct. We find that rationale formats significantly affect how easy it is (1) for users to give feedback for rationales, and (2) for models to subsequently execute this feedback. In addition, formats with attributions to the context and in-depth reasoning significantly enhance user-reported understanding and trust of model outputs.