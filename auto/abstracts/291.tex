We investigate security concerns of the emergent instruction tuning paradigm, that models are trained on crowdsourced datasets with task instructions to achieve superior performance. Our studies demonstrate that an attacker can inject backdoors by issuing very few malicious instructions (\textasciitilde{}1000 tokens) and control model behavior through data poisoning, without even the need to modify data instances or labels themselves. Through such instruction attacks, the attacker can achieve over 90\% attack success rate across four commonly used NLP datasets. As an empirical study on instruction attacks, we systematically evaluated unique perspectives of instruction attacks, such as poison transfer where poisoned models can transfer to 15 diverse generative datasets in a zero-shot manner; instruction transfer where attackers can directly apply poisoned instruction on many other datasets; and poison resistance to continual finetuning. Lastly, we show that RLHF and clean demonstrations might mitigate such backdoors to some degree. These findings highlight the need for more robust defenses against poisoning attacks in instruction-tuning models and underscore the importance of ensuring data quality in instruction crowdsourcing.