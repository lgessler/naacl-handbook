Named entity recognition is a key component of Information Extraction (IE), particularly in scientific domains such as biomedicine and chemistry, where large language models (LLMs), e.g., ChatGPT, fall short. We investigate the applicability of transfer learning for enhancing a named entity recognition model trained in the biomedical domain (the source domain) to be used in the chemical domain (the target domain). A common practice for training such a model in a few-shot learning setting is to pretrain the model on the labeled source data, and then, to finetune it on a hand-full of labeled target examples. In our experiments, we observed that such a model is prone to mislabeling the source entities, which can often appear in the text, as the target entities. To alleviate this problem, we propose a model to transfer the knowledge from the source domain to the target domain, but, at the same time, to project the source entities and target entities into separate regions of the feature space. This diminishes the risk of mislabeling the source entities as the target entities. Our model consists of two stages: 1) entity grouping in the source domain, which incorporates knowledge from annotated events to establish relations between entities, and 2) entity discrimination in the target domain, which relies on pseudo labeling and contrastive learning to enhance discrimination between the entities in the two domains. We conduct our extensive experiments across three source and three target datasets, demonstrating that our method outperforms the baselines by up to 5\textbackslash{}\% absolute value. Code, data, and resources are publicly available for research purposes: https://github.com/Lhtie/Bio-Domain-Transfer .