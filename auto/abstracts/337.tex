Humans frequently experience emotions. When emotions arise, they affect not only our mental state but can also change our physical state. For example, we often open our eyes wide when we are surprised, or clap our hands when we feel excited. Physical manifestations of emotions are referred to as embodied emotion in the psychology literature. From an NLP perspective, recognizing descriptions of physical movements or physiological responses associated with emotions is a type of implicit emotion recognition. Our work introduces a new task of recognizing expressions of embodied emotion in natural language. We create a  dataset of sentences that contains 7,300 body part mentions with human annotations for embodied emotion. We develop a classification model for this task and present two methods to acquire weakly labeled instances  of embodied emotion by extracting emotional manner expressions and by prompting a language model. Our experiments show that the weakly labeled data can  train an effective classification model without  gold data, and can  also improve performance when combined with gold data. Our dataset is publicly available at https://github.com/yyzhuang1991/Embodied-Emotions.