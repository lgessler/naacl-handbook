Recent literature has suggested the potential of using large language models (LLMs) to make classifications for tabular tasks. However, LLMs have been shown to exhibit harmful social biases that reflect the stereotypes and inequalities present in society. To this end, as well as the widespread use of tabular data in many high-stake applications, it is important to explore the following questions: what sources of information do LLMs draw upon when making classifications for tabular tasks; whether and to what extent are LLM classifications for tabular data influenced by social biases and stereotypes; and what are the consequential implications for fairness? Through a series of experiments, we delve into these questions and show that LLMs tend to inherit social biases from their training data which significantly impact their fairness in tabular classification tasks. Furthermore, our investigations show that in the context of bias mitigation, though in-context learning and finetuning have a moderate effect, the fairness metric gap between different subgroups is still larger than that in traditional machine learning models, such as Random Forest and shallow Neural Networks. This observation emphasizes that the social biases are inherent within the LLMs themselves and inherited from their pretraining corpus, not only from the downstream task datasets. Besides, we demonstrate that label-flipping of in-context examples can significantly reduce biases, further highlighting the presence of inherent bias within LLMs.