Metaphorical language is a pivotal element in the realm of political framing. Existing work from linguistics and the social sciences provides compelling evidence regarding the distinctiveness of conceptual framing for political ideology perspectives. However, the nature and utilization of metaphors and the effect on audiences of different political ideologies within political discourses are hardly explored. To enable research in this direction, in this work we create a dataset, originally based on news editorials and labeled with their persuasive effects on liberals and conservatives and extend it with annotations pertaining to metaphorical usage of language. To that end, first, we identify all single metaphors and composite metaphors. Secondly, we provide annotations of the source and target domains for each metaphor. As a result, our corpus consists of 300 news editorials annotated with spans of texts containing metaphors and the corresponding domains of which these metaphors draw from. Our analysis shows that liberal readers are affected by metaphors, whereas conservatives are resistant to them. Both ideologies are affected differently based on the metaphor source and target category. For example, liberals are affected by metaphors in the Darkness \& Light (e.g., death) source domains, where as the source domain of Nature affects conservatives more significantly.