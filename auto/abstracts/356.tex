Spanish is one of the most widespread languages: the official language in 20 countries and the second most-spoken native language. Its contact with other languages across different regions and the rich regional and cultural diversity has produced varieties which divert from each other, particularly in terms of lexicon. Still, available corpora, and models trained upon them, generally treat Spanish as one monolithic language, which dampers prediction and generation power when dealing with different varieties. To alleviate the situation, we compile and curate datasets in the different varieties of Spanish around the world at an unprecedented scale and create the CEREAL corpus. With such a resource at hand, we perform a stylistic analysis to identify and characterise varietal differences. We implement a classifier specially designed to deal with long documents and identify Spanish varieties (and therefore expand CEREAL further). We produce varietal-specific embeddings, and analyse the cultural differences that they encode. We make data, code and models publicly available.