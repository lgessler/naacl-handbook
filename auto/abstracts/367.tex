Traditionally, natural language processing (NLP) models often use a rich set of features created by linguistic expertise, such as semantic representations. However, in the era of large language models (LLMs), more and more tasks are turned into generic, end-to-end sequence generation problems. In this paper, we investigate the question: what is the role of semantic representations in the era of LLMs? Specifically, we investigate the effect of Abstract Meaning Representation (AMR) across five diverse NLP tasks. We propose an AMR-driven chain-of-thought prompting method, which we call AMRCOT, and find that it generally hurts performance more than it helps. To investigate what AMR may have to offer on these tasks, we conduct a series of analysis experiments. We find that it is difficult to predict which input examples AMR may help or hurt on, but errors tend to arise with multi-word expressions, named entities, and in the final inference step where the LLM must connect its reasoning over the AMR to its prediction. We recommend focusing on these areas for future work in semantic representations for LLMs. Our code: https://github.com/causalNLP/amr\_llm