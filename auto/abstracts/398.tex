Despite remarkable advancements in mitigating hallucinations in large language models (LLMs) by retrieval augmentation, it remains challenging to measure the reliability of LLMs using static question-answering (QA) data. Specifically, given the potential of data contamination (e.g., leading to memorization), good static benchmark performance does not ensure that model can reliably use the provided evidence for responding, which is essential to avoid hallucination when the required knowledge is new or private. Inspired by adversarial machine learning, we investigate the feasibility of automatically perturbing existing static one for dynamic evaluation. Specifically, this paper presents ReEval, an LLM-based framework using prompt chaining to perturb the original evidence for generating new test cases for evaluating the LLMs’ reliability in using new evidence for answering. We implement ReEval using ChatGPT and evaluate the resulting variants of two popular open-domain QA datasets on a collection of LLMs under various prompting settings. Our generated data is human-readable and useful to trigger hallucination in LLM. Accurate models on static data are observed to produce unsupported answers from the perturbed evidence, with pronounced accuracy drops across LLMs including GPT-4. We find that our adversarial examples are transferable across all considered LLMs. The examples generated by a small model can be used to evaluate a much larger model, making our approach cost-effective.