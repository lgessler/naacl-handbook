A wave of new task-based virtual assistants has been fueled by increasingly powerful large language models (LLMs), such as GPT-4 (OpenAI, 2023). A major challenge in deploying LLM-based virtual conversational assistants in real world settings is ensuring they operate within what is admissible for the task. To overcome this challenge, the designers of these virtual assistants rely on an independent guardrail system that verifies the virtual assistant’s output aligns with the constraints required for the task. However, relying on commonly used, prompt-based guardrails can be difficult to engineer correctly and comprehensively. To address these challenges, we propose CONSCENDI. We use CONSCENDI to exhaustively generate training data with two key LLM-powered components: scenario-augmented generation and contrastive training examples. When generating conversational data, we generate a set of rule-breaking scenarios, which enumerate a diverse set of high-level ways a rule can be violated. This scenario-guided approach produces a diverse training set and provides chatbot designers greater control. To generate contrastive examples, we prompt the LLM to alter conversations with violations into acceptable conversations to enable fine-grained distinctions. We then use this data, generated by CONSCENDI, to train a smaller model. We find that CONSCENDI results in guardrail models that improve over baselines in multiple dialogue domains.