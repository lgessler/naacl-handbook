Bias in  reporting can influence the public's opinion on relevant societal issues. Examples include informational bias (selective presentation of content) and lexical bias (specific framing of content through linguistic choices). The recognition of media bias is arguably an area where NLP can contribute to the "social good". Traditional NLP models have shown good performance in classifying media bias, but  require careful model design and extensive tuning. In this paper, we ask how well prompting of large language models can recognize media bias. Through an extensive empirical study including a wide selection of pre-trained models, we find that prompt-based techniques can deliver comparable performance to traditional models with greatly reduced effort and that, similar to traditional models, the availability of context substantially improves results. We further show that larger models can leverage different kinds of context simultaneously, obtaining further performance improvements.