The Uniform Information Density (UID) principle posits that humans prefer to spread information evenly during language production. We examine if this UID principle can help capture differences between Large Language Models (LLMs)-generated and human-generated texts. We propose GPT-who, the first psycholinguistically-inspired domain-agnostic statistical detector. This detector employs UID-based features to model the unique statistical signature of each LLM and human author for accurate detection.  We evaluate our method using 4 large-scale benchmark datasets and find that GPT-who outperforms state-of-the-art detectors (both statistical- \textbackslash{}\& non-statistical) such as GLTR, GPTZero, DetectGPT, OpenAI detector, and ZeroGPT by over $20$\% across domains. In addition to better performance,  it is computationally inexpensive and utilizes an interpretable representation of text articles. We find that GPT-who can distinguish texts generated by very sophisticated LLMs, even when the overlying text is indiscernible. UID-based measures for all datasets and code are available at https://github.com/saranya-venkatraman/gpt-who.