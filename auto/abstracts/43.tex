Large Language Models (LLMs) have shown to be capable of various tasks, yet their capability in interpreting and reasoning over tabular data remains an underexplored area. In this context, this study investigates from three core perspectives: the robustness of LLMs to structural perturbations in tables, the comparative analysis of textual and symbolic reasoning on tables, and the potential of boosting model performance through the aggregation of multiple reasoning pathways. We discover that structural variance of tables presenting the same content reveals a notable performance decline, particularly in symbolic reasoning tasks. This prompts the proposal of a method for table structure normalization. Moreover, textual reasoning slightly edges out symbolic reasoning, and a detailed error analysis reveals that each exhibits different strengths depending on the specific tasks. Notably, the aggregation of textual and symbolic reasoning pathways, bolstered by a mix self-consistency mechanism, resulted in achieving SOTA performance, with an accuracy of 73.6\% on WikiTableQuestions, representing a substantial advancement over previous existing table processing paradigms of LLMs.