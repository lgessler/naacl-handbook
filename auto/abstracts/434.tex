Interpretability research has shown that self-supervised Spoken Language Models (SLMs) encode a wide variety of features in human speech from the acoustic, phonetic, phonological, syntactic and semantic levels, to speaker characteristics. The bulk of prior research on representations of phonology has focused on segmental features such as phonemes; the encoding of suprasegmental phonology (such as tone and stress patterns) in SLMs is not yet well understood. Tone is a suprasegmental feature that is present in more than half of the world's languages. This paper aims to analyze the tone encoding capabilities of SLMs, using Mandarin and Vietnamese as case studies. We show that SLMs encode lexical tone to a significant degree even when they are trained on data from non-tonal languages. We further find that SLMs behave similarly to native and non-native human participants in tone and consonant perception studies, but they do not follow the same developmental trajectory.