Single document news summarization has seen substantial progress on faithfulness in recent years, driven by research on the evaluation of factual consistency, or hallucinations. We ask whether these advances carry over to other text summarization domains. We propose a new evaluation benchmark on topic-focused dialogue summarization, generated by LLMs of varying sizes. We provide binary sentence- level human annotations of the factual consistency of these summaries along with detailed explanations of factually inconsistent sentences. Our analysis shows that existing LLMs hallucinate significant amounts of factual errors in the dialogue domain, regardless of the model’s size. On the other hand, when LLMs, including GPT-4, serve as binary factual evaluators, they perform poorly and can be outperformed by prevailing state-of-the-art specialized factuality evaluation metrics. Finally, we conducted an analysis of hallucination types with a curated error taxonomy. We find that there are diverse errors and error distributions in model-generated summaries and that non-LLM based metrics can capture all error types better than LLM-based evaluators.