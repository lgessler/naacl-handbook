Gender-inclusive NLP research has documented the harmful limitations of gender binary-centric large language models (LLM), such as the inability to correctly use gender-diverse English neopronouns (e.g., xe, zir, fae). While data scarcity is a known culprit, the precise mechanisms through which scarcity affects this behavior remain underexplored. We discover LLM misgendering is significantly influenced by Byte-Pair Encoding (BPE) tokenization, the tokenizer powering many popular LLMs. Unlike binary pronouns, BPE overfragments neopronouns, a direct consequence of data scarcity during tokenizer training. This disparate tokenization mirrors tokenizer limitations observed in multilingual and low-resource NLP, unlocking new misgendering mitigation strategies. We propose two techniques: (1) pronoun tokenization parity, a method to enforce consistent tokenization across gendered pronouns, and (2) utilizing pre-existing LLM pronoun knowledge to improve neopronoun proficiency. Our proposed methods outperform finetuning with standard BPE, improving neopronoun accuracy from 14.1\% to 58.4\%. Our paper is the first to link LLM misgendering to tokenization and deficient neopronoun grammar, indicating that LLMs unable to correctly treat neopronouns as pronouns are more prone to misgender.