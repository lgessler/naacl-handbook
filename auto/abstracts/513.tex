Automatic speech recognition (ASR) systems, increasingly prevalent in education, healthcare, employment, and mobile technology, face significant challenges in inclusivity, particularly for the 80 million-strong global community of people who stutter. These systems often fail to accurately interpret speech patterns deviating from typical fluency, leading to critical usability issues and misinterpretations. This study evaluates six leading ASRs, analyzing their performance on both a real-world dataset of speech samples from individuals who stutter and a synthetic dataset derived from the widely-used LibriSpeech benchmark. The synthetic dataset, uniquely designed to incorporate various stuttering events, enables an in-depth analysis of each ASR's handling of disfluent speech. Our comprehensive assessment includes metrics such as word error rate (WER), character error rate (CER), and semantic accuracy of the transcripts. The results reveal a consistent and statistically significant accuracy bias across all ASRs against disfluent speech, manifesting in significant syntactical and semantic inaccuracies in transcriptions. These findings highlight a critical gap in current ASR technologies, underscoring the need for effective bias mitigation strategies. Addressing this bias is imperative not only to improve the technology's usability for people who stutter but also to ensure their equitable and inclusive participation in the rapidly evolving digital landscape.