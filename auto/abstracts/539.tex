In-context learning (ICL) is an appealing approach for semantic parsing due to its few-shot nature and improved generalization. However, learning to parse to rare domain-specific languages (DSLs) from just a few demonstrations is challenging, limiting the performance of even the most capable LLMs. In this work, we show how pre-existing coding abilities of LLMs can be leveraged for semantic parsing by (1) using general-purpose programming languages such as Python instead of DSLs and (2) augmenting prompts with a structured domain description that includes, e.g., the available classes and functions. We show that both these changes significantly improve accuracy across three popular datasets; combined, they lead to dramatic improvements (e.g., 7.9\% to 66.5\% on SMCalFlow compositional split) and can substantially improve compositional generalization, nearly closing the performance gap between easier i.i.d. and harder compositional splits. Finally, comparisons across multiple PLs and DSL variations suggest that the similarity of a target language to general-purpose code is more important than prevalence in pretraining corpora. Our findings provide an improved methodology for building semantic parsers in the modern context of ICL with LLMs.