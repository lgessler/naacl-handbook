Large language models have become valuable tools for data augmentation in scenarios with limited data availability, as they can generate synthetic data resembling real-world data. However, their generative performance depends on the quality of the prompt used to instruct the model. Prompt engineering that relies on hand-crafted strategies or requires domain experts to adjust the prompt often yields suboptimal results. In this paper we present SAPE, a Spanish Adaptive Prompt Engineering method utilizing genetic algorithms for prompt generation and selection. Our evaluation of SAPE focuses on a generative task that involves the creation of Spanish therapy transcripts, a type of data that is challenging to collect due to the fact that it typically includes  protected health information. Through human evaluations conducted by mental health professionals, our results show that SAPE produces Spanish counselling transcripts that more closely resemble authentic therapy transcripts compared to other prompt engineering techniques that are based on Reflexion and Chain-of-Thought.