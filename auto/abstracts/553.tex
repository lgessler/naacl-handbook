The self-improving ability of large language models (LLMs), enabled by prompting them to analyze and revise their own outputs, has garnered significant interest in recent research. However, this ability has been shown to be absent and difficult to learn for smaller models, thus widening the performance gap between state-of-the-art LLMs and more cost-effective and faster ones. To reduce this gap, we introduce TriPosT, a training algorithm that endows smaller models with such self-improvement ability, and show that our approach can improve LLaMA-7B's performance on math and reasoning tasks by up to 7.13\%. In contrast to prior work, we achieve this by using the smaller model to interact with LLMs to collect feedback and improvements on *its own generations*. We then replay this experience to train the small model. Our experiments on four math and reasoning datasets show that the interactive experience of learning from and correcting its *own* mistakes is crucial for small models to improve their performance.