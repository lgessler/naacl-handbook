We explore the creative problem-solving capabilities of modern LLMs in a novel constrained setting. To this end, we create MACGYVER, an automatically generated dataset consisting of over 1,600 real-world problems deliberately designed to trigger innovative usage of objects and necessitate out-of-the-box thinking. We then present our collection to both LLMs and humans to compare and contrast their problem-solving abilities. MACGYVER is challenging for both groups, but in unique and complementary ways. For instance, humans excel in tasks they are familiar with but struggle with domain-specific knowledge, leading to a higher variance. In contrast, LLMs, exposed to a variety of specialized knowledge, attempt broader problems but fail by proposing physically-infeasible actions. Finally, we provide a detailed error analysis of LLMs, and demonstrate the potential of enhancing their problem-solving ability with novel prompting techniques such as iterative step-wise reflection and divergent-convergent thinking. This work (1) introduces a fresh arena for intelligent agents focusing on intricate aspects of physical reasoning, planning, and unconventional thinking, which supplements the existing spectrum of machine intelligence; and (2) provides insight into the constrained problem-solving capabilities of both humans and AI.