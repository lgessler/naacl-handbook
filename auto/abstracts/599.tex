Social media is a valuable data source for exploring mental health issues.  However, previous studies have predominantly focused on the semantic content of these posts, overlooking the importance of their temporal attributes, as well as the evolving nature of mental disorders and symptoms. In this paper, we study the causality between psychiatric symptoms and life events, as well as among different symptoms from social media posts, which leads to better understanding of the underlying mechanisms of mental disorders. By applying these extracted causality features to tasks such as diagnosis point detection and early risk detection of depression, we notice considerable performance enhancement. This indicates that causality information extracted from social media data can boost the efficacy of mental disorder diagnosis and treatment planning.