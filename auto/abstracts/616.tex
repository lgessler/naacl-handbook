While large language models (LMs) demonstrate remarkable performance, they encounter challenges in providing accurate responses when queried for information beyond their pre-trained memorization. Although augmenting them with relevant external information can mitigate these issues, failure to consider the necessity of retrieval may adversely affect overall performance. Previous research has primarily focused on examining how entities influence retrieval models and knowledge recall in LMs, leaving other aspects relatively unexplored. In this work, our goal is to offer a more detailed, fact-centric analysis by exploring the effects of combinations of entities and relations. To facilitate this, we construct a new question answering (QA) dataset called WiTQA (Wikipedia Triple Question Answers). This dataset includes questions about entities and relations of various popularity levels, each accompanied by a supporting passage. Our extensive experiments with diverse LMs and retrievers reveal when retrieval does not consistently enhance LMs from the viewpoints of fact-centric popularity. Confirming earlier findings, we observe that larger LMs excel in recalling popular facts. However, they notably encounter difficulty with infrequent entity-relation pairs compared to retrievers. Interestingly, they can effectively retain popular relations of less common entities. We demonstrate the efficacy of our finer-grained metric and insights through an adaptive retrieval system that selectively employs retrieval and recall based on the frequencies of entities and relations in the question.