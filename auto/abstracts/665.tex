In-context learning (ICL) has become one of the most popular learning paradigms. While there is a growing body of literature focusing on prompt engineering, there is a lack of systematic analysis comparing the effects of prompt techniques across different models and tasks. To address this, we present a comprehensive prompt analysis based on sensitivity. Our analysis reveals that sensitivity is an unsupervised proxy for model performance, as it exhibits a strong negative correlation with accuracy. We use gradient-based saliency scores to empirically demonstrate how different prompts affect the relevance of input tokens to the output, resulting in different levels of sensitivity. Furthermore, we introduce sensitivity-aware decoding which incorporates sensitivity estimation as a penalty term in the standard greedy decoding. We show that this approach is particularly helpful when information in the input is scarce. Our work provides a fresh perspective on the analysis of prompts, and contributes to a better understanding of the mechanism of ICL.