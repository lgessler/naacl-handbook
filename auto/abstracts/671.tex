Recently, one popular alternative in Multilingual NMT (MNMT) is modularized MNMT that has both language-specific encoders and decoders. However, due to the absence of layer-sharing, the modularized MNMT failed to produce satisfactory language-independent (Interlingua) features, leading to performance degradation in zero-shot translation. To address this issue, a solution was proposed to share the top of language-specific encoder layers, enabling the successful generation of interlingua features.  Nonetheless, it should be noted that this sharing structure does not guarantee the explicit propagation of language-specific features to their respective language-specific decoders.  Consequently, to overcome this challenge, we present our modularized MNMT approach, where a modularized encoder is divided into three distinct encoder modules based on different sharing criteria: (1) source language-specific ($Enc_{s}$); (2) universal ($Enc_{all}$); (3) target language-specific ($Enc_{t}$). By employing these sharing strategies, $Enc_{all}$ propagates the interlingua features, after which $Enc_{t}$ propagates the target language-specific features to the language-specific decoders. Additionally, we suggest the Denoising Bi-path Autoencoder (DBAE) to fortify the Denoising Autoencoder (DAE) by leveraging $Enc_{t}$. For experimental purposes, our training corpus comprises both En-to-Any and Any-to-En directions. We adjust the size of our corpus to simulate both balanced and unbalanced settings. Our method demonstrates an improved average BLEU score by "+2.90" in En-to-Any directions and by "+3.06" in zero-shot compared to other MNMT baselines.