Text classification systems have continuously improved in performance over the years. How- ever, nearly all current SOTA classifiers have a similar shortcoming, they process text in a hor- izontal manner. Vertically written words will not be recognized by a classifier. In contrast, humans are easily able to recognize and read words written both horizontally and vertically. Hence, a human adversary could write problem- atic words vertically and the meaning would still be preserved to other humans. We simulate such an attack, VertAttack. VertAttack identifies which words a classifier is reliant on and then rewrites those words vertically. We find that VertAttack is able to greatly drop the accuracy of 4 different transformer models on 5 datasets. For example, on the SST2 dataset, VertAttack is able to drop RoBERTa’s accuracy from 94 to 13\%. Furthermore, since VertAttack does not replace the word, meaning is easily preserved. We verify this via a human study and find that crowdworkers are able to correctly label 77\% perturbed texts perturbed, compared to 81\% of the original texts. We believe VertAttack offers a look into how humans might circumvent clas- sifiers in the future and thus inspire a look into more robust algorithms.