We study how to apply large language models to write grounded and organized long-form articles from scratch, with comparable breadth and depth to Wikipedia pages. This underexplored problem poses new challenges at the pre-writing stage, including how to research the topic and prepare an outline prior to writing. We propose STORM, a writing system for the Synthesis of Topic Outlines through Retrieval and Multi-perspective Question Asking. STORM models the pre-writing stage by (1) discovering diverse perspectives in researching the given topic, (2) simulating conversations where writers carrying different perspectives pose questions to a topic expert grounded on trusted Internet sources, (3) curating the collected information to create an outline. For evaluation, we curate FreshWiki, a dataset of recent high-quality Wikipedia articles, and formulate outline assessments to evaluate the pre-writing stage. We further gather feedback from experienced Wikipedia editors. Compared to articles generated by an outline-driven retrieval-augmented baseline, more of STORM’s articles are deemed to be organized (by a 25\% absolute increase) and broad in coverage (by 10\%). The expert feedback also helps identify new challenges for generating grounded long articles, such as source bias transfer and over-association of unrelated facts.