In recent times, large language models (LLMs) have shown impressive performance on various document-level tasks such as document classification, summarization, and question-answering. However, research on understanding their capabilities on the task of self-contradictions in long documents has been very limited. In this work, we introduce ContraDoc, the first human-annotated dataset to study self-contradictions in long documents across multiple domains, varying document lengths, self-contradiction types, and appearance scope. We then analyze the current capabilities of four state-of-the-art open-source and commercially available LLMs: GPT3.5, GPT4, PaLM2, and LLaMAv2 on this dataset.  While GPT4 performs the best and can outperform humans on this task, we find that it is still unreliable and struggles with self-contradictions that require more nuance and context. We release the dataset and all the code associated with the experiments.