Extensive efforts in the past have been directed toward the development of summarization datasets. However, a predominant number of these resources have been (semi)-automatically generated, typically through web data crawling. This resulted in subpar resources for training and evaluating summarization systems, a quality compromise that is arguably due to the substantial costs associated with generating ground-truth summaries, particularly for diverse languages and specialized domains. To address this issue, we present ACLSum, a novel summarization dataset carefully crafted and evaluated by domain experts. In contrast to previous datasets, ACLSum facilitates multi-aspect summarization of scientific papers, covering challenges, approaches, and outcomes in depth. Through extensive experiments, we evaluate the quality of our resource and the performance of models based on pretrained language models (PLMs) and state-of-the-art large language models (LLMs). Additionally, we explore the effectiveness of extract-then-abstract versus abstractive end-to-end summarization within the scholarly domain on the basis of automatically discovered aspects. While the former performs comparably well to the end-to-end approach with pretrained language models regardless of the potential error propagation issue, the prompting-based approach with LLMs shows a limitation in extracting sentences from source documents.