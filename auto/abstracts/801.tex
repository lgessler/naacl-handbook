The paper focuses on the marginalization of indigenous language communities in the face of rapid technological advancements. We highlight the cultural richness of these languages and the risk they face of being overlooked in the realm of Natural Language Processing (NLP). We aim to bridge the gap between these communities and researchers, emphasizing the need for inclusive technological advancements that respect indigenous community perspectives. We show the NLP progress of indigenous Latin American languages and the survey that covers the status of indigenous languages in Latin America, their representation in NLP, and the challenges and innovations required for their preservation and development. The paper contributes to the current literature in understanding the need and progress of NLP for indigenous communities of Latin America, specifically low-resource and indigenous communities in general.