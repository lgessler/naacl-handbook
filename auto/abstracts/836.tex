Social media, originally meant for peaceful communication, now faces issues with hate speech. Detecting hate speech from social media in Indian languages with linguistic diversity and cultural nuances presents a complex and challenging task. Furthermore, traditional methods involve sharing of users' sensitive data with a server for model training making it undesirable and involving potential risk to their privacy remained under-studied. In this paper, we combined various low-resource language datasets and propose MultiFED, a federated approach that performs effectively to detect hate speech. MultiFED utilizes continuous adaptation and fine-tuning to aid generalization using subsets of multilingual data overcoming the limitations of data scarcity. Extensive experiments are conducted on 13 Indic datasets across five different pre-trained models. The results show that MultiFED outperforms the state-of-the-art baselines by 8\textbackslash{}\% (approx.) in terms of Accuracy and by 12\textbackslash{}\% (approx.) in terms of F-Score.