Questions posed by information-seeking users often contain implicit false or potentially harmful assumptions. In a high-risk domain such as maternal and infant health, a question-answering system must recognize these pragmatic constraints and go beyond simply answering user questions, examining them in context to respond helpfully. To achieve this, we study assumptions and implications, or pragmatic inferences, made when mothers ask questions about pregnancy and infant care by collecting a dataset of 2,727 inferences from 500 questions across three diverse sources. We study how health experts naturally address these inferences when writing answers, and  illustrate that informing  existing QA pipelines with pragmatic inferences produces responses that are more complete, mitigating the propagation of harmful beliefs.