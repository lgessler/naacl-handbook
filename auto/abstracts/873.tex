As large language models (LLMs) have increased in their capabilities, so does their potential for dual use. To reduce harmful outputs, produces and vendors of LLMs have used reinforcement learning with human feedback (RLHF). In tandem, LLM vendors have been increasingly enabling fine-tuning of their most powerful models. However, concurrent work has shown that fine-tuning can remove RLHF protections. We may expect that the most powerful models currently available (GPT-4) are less susceptible to fine-tuning attacks.  In this work, we show the contrary: fine-tuning allows attackers to remove RLHF protections with as few as 340 examples and a 95\textbackslash\{\}\% success rate. These training examples can be automatically generated with weaker models. We further show that removing RLHF protections does not decrease usefulness on non-censored outputs, providing evidence that our fine-tuning strategy does not decrease usefulness despite using weaker models to generate training data. Our results show the need for further research on protections on LLMs.