In the realm of Large Language Models (LLMs), the balance between instruction data quality and quantity is a focal point. Recognizing this, we introduce a self-guided methodology for LLMs to autonomously discern and select cherry samples from open-source datasets, effectively minimizing manual curation and potential cost for instruction tuning an LLM. Our key innovation, the Instruction-Following Difficulty (IFD) metric, emerges as a pivotal metric to identify discrepancies between a model's expected responses and its intrinsic generation capability. Through the application of IFD, cherry samples can be pinpointed, leading to a marked uptick in model training efficiency. Empirical validations on datasets like Alpaca and WizardLM underpin our findings; with a mere 10\textbackslash\{\}\% of original data input, our strategy showcases improved results. This synthesis of self-guided cherry-picking and the IFD metric signifies a transformative leap in the instruction tuning of LLMs, promising both efficiency and resource-conscious advancements. Codes, data, and models are available.