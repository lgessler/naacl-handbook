Recent advances in named entity recognition (NER) have pushed the boundary of the task to incorporate visual signals, leading to many variants, including multi-modal NER (MNER) or grounded MNER (GMNER).  A key challenge to these tasks is that the model should be able to generalize to the entities unseen during the training, and should be able to handle the training samples with noisy annotations. To address this obstacle, we propose SCANNER (Span CANdidate detection and recognition for NER), a model capable of effectively handling all three NER variants. SCANNER is a two-stage structure; we extract entity candidates in the first stage and use it as a query to get knowledge, effectively pulling knowledge from various sources. We can boost our performance by utilizing this entity-centric extracted knowledge to address unseen entities. Furthermore, to tackle the challenges arising from noisy annotations in NER datasets, we introduce a novel self-distillation method, enhancing the robustness and accuracy of our model in processing training data with inherent uncertainties. Our approach demonstrates competitive performance on the NER benchmark and surpasses existing methods on both MNER and GMNER benchmarks. Further analysis shows that the proposed distillation and knowledge utilization methods improve the performance of our model on various benchmarks.