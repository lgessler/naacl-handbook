Cross-lingual continual learning aims to continuously fine-tune a downstream model on emerging data from new languages. One major challenge in cross-lingual continual learning is catastrophic forgetting: a stability-plasticity dilemma, where performance on previously seen languages decreases as the model learns to transfer to new languages. Experience replay, which revisits data from a fixed-size memory of old languages while training on new ones, is among the most successful approaches for solving this dilemma. Faced with the challenge of dynamically storing the memory with high-quality examples while complying with its fixed size limitations, we consider Leitner queuing, a human-inspired spaced-repetition technique, to determine what should be replayed at each phase of learning. Via a controlled set of quantitative and qualitative analyses across different memory strategies, we show that, just like humans, carefully picking informative examples to be prioritized in cross-lingual memory replay helps tame the stability-plasticity dilemma. Compared to vanilla and strong memory replay baselines, our Leitner-guided approach significantly and consistently decreases forgetting while maintaining accuracy across natural language understanding tasks, language orders, and languages.