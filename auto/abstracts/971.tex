Several Natural Language Understanding (NLU) tasks focus on linking text to explicit knowledge, including Word Sense Disambiguation, Semantic Role Labeling, Semantic Parsing, and Relation Extraction. In addition to the importance of connecting raw text with explicit knowledge bases, the integration of such carefully curated knowledge into deep learning models has been shown to be beneficial across a diverse range of applications, including Language Modeling and Machine Translation. Nevertheless, the scarcity of semantically-annotated corpora across various tasks and languages limits the potential advantages significantly. To address this issue, we put forward MOSAICo, the first endeavor aimed at equipping the research community with the key ingredients to model explicit semantic knowledge at a large scale, providing hundreds of millions of silver yet high-quality annotations for four NLU tasks across five languages. We describe the creation process of MOSAICo, demonstrate its quality and variety, and analyze the interplay between different types of semantic information. MOSAICo, available at https://github.com/SapienzaNLP/mosaico, aims to drop the requirement of closed, licensed datasets and represents a step towards a level playing field across languages and tasks in NLU.