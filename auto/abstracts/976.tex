We introduce BUST, a comprehensive benchmark designed to evaluate detectors of texts generated by instruction-tuned large language models (LLMs). Unlike previous benchmarks, our focus lies on evaluating the performance of detector systems, acknowledging the inevitable influence of the underlying tasks and different LLM generators. Our benchmark dataset consists of 25K texts from humans and 7 LLMs responding to instructions across 10 tasks from 3 diverse sources. Using the benchmark, we evaluated 5 detectors and found substantial performance variance across tasks. A meta-analysis of the dataset characteristics was conducted to guide the examination of detector performance. The dataset was analyzed using diverse metrics assessing linguistic features like fluency and coherence, readability scores, and writer attitudes, such as emotions, convincingness, and persuasiveness. Features impacting detector performance were investigated with surrogate models, revealing emotional content in texts enhanced some detectors, yet the most effective detector demonstrated consistent performance, irrespective of writer’s attitudes and text styles. Our approach focused on investigating relationships between the detectors' performance and two key factors: text characteristics and LLM generators. We believe BUST will provide valuable insights into selecting detectors tailored to specific text styles and tasks and facilitate a more practical and in-depth investigation of detection systems for LLM-generated text.