Conversational search aims to retrieve passages containing essential information to answer queries in a multi-turn conversation.  In conversational search, reformulating context-dependent conversational queries into stand-alone forms is imperative to effectively utilize off-the-shelf retrievers.  Previous methodologies for conversational query reformulation frequently depend on human-annotated rewrites. However, these manually crafted queries often result in sub-optimal retrieval performance and require high collection costs. To address these challenges, we propose **Iter**ative **C**onversational **Q**uery **R**eformulation (**IterCQR**), a methodology that conducts query reformulation without relying on human rewrites.  IterCQR iteratively trains the conversational query reformulation (CQR) model by directly leveraging information retrieval (IR) signals as a reward. Our IterCQR training guides the CQR model such that generated queries contain necessary information from the previous dialogue context. Our proposed method shows state-of-the-art performance on two widely-used datasets, demonstrating its effectiveness on both sparse and dense retrievers.  Moreover, IterCQR exhibits superior performance in challenging settings such as generalization on unseen datasets and low-resource scenarios.