Large Language Models (LLMs) are capable of successfully performing many language processing tasks zero-shot (without training data). If zero-shot LLMs can also reliably classify and explain social phenomena like persuasiveness and political ideology, then LLMs could augment the Computational Social Science (CSS) pipeline in important ways. This work provides a road map for using LLMs as CSS tools. Towards this end, we contribute a set of prompting best practices and an extensive evaluation pipeline to measure the zero-shot performance of 13 language models on 25 representative English CSS benchmarks. On taxonomic labeling tasks (classification), LLMs fail to outperform the best fine-tuned models but still achieve fair levels of agreement with humans. On free-form coding tasks (generation), LLMs produce explanations that often exceed the quality of crowdworkers' gold references. We conclude that the performance of today's LLMs can augment the CSS research pipeline in two ways: (1) serving as zero-shot data annotators on human annotation teams, and (2) bootstrapping challenging creative generation tasks (e.g., explaining the underlying attributes of a text). In summary, LLMs are posed to meaningfully participate in social science analysis in partnership with humans.