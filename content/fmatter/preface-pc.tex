\section{Message from the Program Chairs}
\setheaders%
    {Message from the Program Chairs}%
    {Message from the Program Chairs}
\thispagestyle{emptyheader}
\setlength{\parskip}{1ex}
\setlength{\parindent}{0pt}

Hi, welcome to the 2024 Annual Conference of the North American Association for Computational Linguistics! NAACL 2024 is a hybrid conference, and we are excited to have attendees and presenters join us both in person in Mexico City and online from all over the world. We are especially thrilled to hold the conference in Mexico City, which was originally planned for NAACL 2021 before COVID-19 required the transition to a virtual meeting. This will be the first NAACL conference in Latin America, and we hope this will contribute to a tradition of broadening access and participation in the greater region. 

\subsubsection*{Special Theme: Languages of Latin America}
Languages are the heart and soul of cultural identity and communication, and nowhere is this more evident than in the vibrant tapestry of Latin America and the Caribbean. With a rich linguistic diversity that spans Spanish, Portuguese, and numerous indigenous languages, the region offers a unique challenge and opportunity for natural language processing researchers. For NAACL 2024, we invited submissions to the special theme track on ``Languages of Latin America''. This track was dedicated to taking stock of past research and developments in the field of natural language processing for languages of Latin America and the Caribbean while charting the course for future investigations. We received 19 submissions to the special theme, of which 10 have been accepted to appear at the conference.

\subsubsection*{Review Process}
NAACL 2024 implemented a two stage review process, where submissions were first sent to ACL Rolling Review (ARR) for reviews by reviewers and for meta-reviews by area chairs, and then sent to a separate NAACL 2024 commitment site for recommendations by senior area chairs and final acceptance decisions by program chairs.

For the ARR submission part of the process, NAACL program chairs coordinated with EACL 2024 and ACL 2024 program chairs to ensure a smooth revise-and-resubmit cycle across the three conferences. We also coordinated across conferences to recruit thousands of new reviewers and hundreds of new area chairs to ARR, resulting in 7344 reviewers and 870 ACs in the 2023 December ARR cycle to which most NAACL 2024 papers were submitted. Overall, the ARR process went mostly smoothly, successfully delivering three reviews and a meta-review for all 2604 papers submitted. Several of the suggestions that NAACL 2024 program chairs collected for improving the process (e.g., better OpenReview integration of the responsible NLP checklist) have already been adopted by ARR for future cycles.

For the NAACL commitment part of the process, NAACL program chairs recruited 73 senior area chairs for the 25 research areas defined by ARR. Senior area chairs made acceptance recommendations for 1140 committed papers based on the papers, reviews, and meta-reviews, and program chairs finalized the recommendations into acceptance decisions.

\subsubsection*{Acceptance Rate}
The acceptance rate calculation follows precedent set by previous conferences that also go through ARR, e.g. NAACL 2022, EACL 2024. The calculation takes into account the multi-stage process of ARR where a paper may get revised in ARR and then later committed to the conference. The denominator includes:
\begin{itemize}
\setlength\parskip{0em}
\setlength\itemsep{0.3em}
\item Papers in the ARR December 2023 cycle that selected NAACL as a preferred venue
\item Papers in the ARR December 2023 cycle that did not select any conference as a preferred venue
\item Papers in the ARR December 2023 cycle that selected another conference, but then committed to NAACL 2024.
\item Papers in the ARR cycles before December 2023 that committed to NAACL 2024.
\end{itemize}

In total, we had 2604 submissions in the ARR December 2023 cycle. Among these, 29 were withdrawn before reviews were released and 115 were desk-rejected. Of the remaining, 2328 had either an unspecified venue or included NAACL as the desired venue. Further, 17 out of the 132 submissions that selected other venues were committed to NAACL. Finally, an additional 89 papers from other cycles were committed. So in total, the denominator for acceptance rate calculation is 2328 + 17 + 89 = 2434. Among these, 1140 papers were officially committed to NAACL, and 565 were accepted. The acceptance rate for Main Conference papers is therefore: 565 / 2434 = 23.2\%. The final Main Conference proceedings included 487 long papers and 75 short papers (3 papers withdrew after the acceptance decision).

Findings papers are those which are not accepted at the Main Conference, but nevertheless have been judged worthy of publication as solid work with sufficient substance, quality and novelty. The next 304 / 2434 = 12.5\% of papers were accepted to NAACL Findings. The final Findings proceedings included 255 long papers and 42 short papers (7 papers withdrew after the acceptance decision).

\subsubsection*{Presentation Format}
At NAACL 2024, we aimed to set all main conference papers on equal ground. All presenters were allowed the same 13 minute video recording on the virtual site, regardless of whether a paper was long or short, whether the presenter decided to attend in-person or virtually, and whether the paper was assigned an oral presentation or a poster presentation.

To ensure there was no prestige associated with getting to present in oral vs. poster format, we tried a new approach to presentation decisions: we assigned them randomly. Specifically, we calculated the counts of papers across research areas, took the square roots of the counts to slightly upweight smaller areas, converted the counts to a distribution, and then randomly sampled 130 orals from the research areas according to the distribution (sampling in blocks of 5 to match the duration of oral sessions at the conference).
Program Format
At NAACL 2024, we aimed to improve both the in-person and virtual experiences. For this, we are implemented the following two actions:

A pre-conference virtual poster session was scheduled for Thursday, June 13, 2024, avoiding conflicts with the conference’s in-person sessions, and including different sessions to accommodate various time zones. The goal of this move was to encourage all attendees, both virtual and in-person, to join the virtual poster session.
Oral presentations were given only to in-person attendees. (Oral presentations were still set to be live-streamed for all virtual attendees). The goal of this move was to avoid Zoom fatigue and encourage more in-person engagement with oral presenters.

The program includes live (and live-streamed) keynotes, plenaries, and panels, more than 100 live (and live-streamed) oral presentations, more than 400 live poster presentations, and more than 200 virtual poster presentations at the pre-conference event. The keynotes cover exciting topics including large language models and indigenous languages (Claudio Pinhanez, IBM Research Brazil) and large language models and neuroscience (Seana Coulson, UCSD), while the panel addresses the important issue of large language models and their impact on education (Victoria Yaneva, National Board of Medical Examiners; Swapna Somasundaran, Educational Testing Service; Karen Matías, Universidad Nacional Autónoma de México; and Ekaterina Kochmar, Mohamed Bin Zayed University of Artificial Intelligence). Other plenaries include the NAACL business meeting and the best paper awards session. The program is rounded out with dedicated sessions during the main conference for industry track, demonstrations track, student research workshop, NAACL Findings papers, and TACL/CL accepted papers.

\subsubsection*{Gratitude}
Conference organization is a team effort. We are very grateful for the support and contributions of many people, including: 

\begin{itemize}
\setlength\parskip{0em}
\setlength\itemsep{0.3em}
\item The General Chair, Katrin Erk
\item The ARR Editors-in-Chief of the Dec 2023 cycle (Mausam, Vincent Ng) and the entire team (Viviane Moreira, Thamar Solorio, Lilja {\O}vrelid, Jun Suzuki, Jonathan Kummerfield)
\item The OpenReview team, especially Harold Rubio 
\item The 73 Senior Area Chairs
\item The 870 Area Chairs and 7344 Reviewers 
\item The best paper committee chairs, Isabelle Augenstein and Manuel Montes y G\'omez
\item The ethics chairs (Cecilia Alm, Diana Galvan Sosa, Anjalie Field, Ameeta Agrawal, Daniel Fried, Mark Yatskar, Maria Antoniak, Alane Suhr) and their team of reviewers
\item The website chairs, Vered Shwartz and Xinya Du
\item The publication chairs, Ryan Cotterell, Maarten Sap, and Lifu Huang, and their team of student helpers
\item The publicity chairs, Ximena Gutierrez-Vasques, Samuel Gonzalez-Lopez, and Najoung Kim
\item The local chair, Hiram Calvo
\item The volunteers chairs, Lucy Lu Wang and Liang Huang
\item The ACL Anthology Director, Matt Post, and his team
\item The Program Chairs of EACL 2024 (Yvette Graham, Matthew Purver) and ACL 2024 (Lun-Wei Ku, Andre Martins, Vivek Srikumar)
\item Damira Mr\v{s}i\'{c} and Underline Team
\item Jenn Rachhford and entire conference support staff
\end{itemize}

We hope you will enjoy NAACL 2024!

\noindent NAACL 2024 Program Chairs \\
Kevin Duh, Johns Hopkins University \\
Helena Gomez, Universidad Nacional Aut\'onoma de M\'exico \\
Steve Bethard, University of Arizona