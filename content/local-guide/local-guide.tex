\chapter{Local Guide}

{\bf\Large Mexico City Essentials: How to Navigate the Chaos}

\section*{Getting Around}

\subsection*{Public Transport}
The network is comprised mainly by:

\begin{itemize}
\setlength\parskip{0em}
\setlength\itemsep{0.3em}
\item Metro (subway)
\item Metrobus (red buses that move in a confined lane)
\item Trolebus (blue trolleybus)
\item RTP  (green buses)
\end{itemize}

In all cases, you can access using a rechargeable card (tarjeta MI). You can buy/recharge this card in any subway station and other places. You  can also recharge through the ``app CDMX'' on your phone.

Remember that Mexico City is one of the most populated cities in the world. Buses and subways can be crowded during rush hour or trapped in traffic jams, there are no fixed schedules.

\subsection*{Uber}

Moving around using Uber is also quite common, the app is available in the city. .

\subsection*{Bicycle}

There is a public bike sharing system called ``Ecobici'', you can borrow a bike but you need to register first.

On Sundays, car traffic is banned on the main avenues (like Reforma), so people can move around the city using a bike or jogging, rolling skating, etc. This is a nice way to get to know the city (from 8:00 to 14:00). Check here for extended info.

\subsection*{Tourism Bus}
There is a touristic bus that covers several routes around the city, it is called the ``turibus''. Check the schedules, prices and where to board here


\section*{General Info}

\subsection*{Tourism}
The conference venue is located in the very city center. There are many sites of interest nearby. For example:

\begin{itemize}
  \setlength\parskip{0em}
  \setlength\itemsep{0.3em}
\item El Zocalo (the main central square)
\item Monumento a la revoluci\'on
\item Museo Templo Mayor (museum with the remains of the ancient city that lies below current central Mexico City)
\item Torre Latinoamericana (a skyscraper with a viewing platform)
\item Edificio de Correos (Historic postal palace)
\item Museo de antropolog\'{\i}a e historia (Museum about the indigenous cultures all over Mexico)
\item Castillo de Chapultepec (a historic castle inside an urban forest)
\item  Art Museums: Museo Nacional de Arte, Museo Soumaya
\item You can take several walking routes through the city center
\end{itemize}

For more info visit this official guide (available in English)

Please note that if you want to travel to some archeological site or town outside the city, you will have to take a foreign bus or rent a car (no trains available)

\subsection*{Safety}
In Mexico City, tap water is usually NOT drinkable (unless the place where you are staying explicitly states otherwise). It is recommended that you buy bottled water.

The city is generally safe and walkable (also public transportation). However, as in other big cities, watch out for your belongings and avoid walking alone late at night. Also, be cautious when crossing avenues or streets with heavy car traffic to prevent accidents.

There is no need to exchange large amounts of money at the airport. You can use your credit/debit cards, make contactless payments, or retrieve cash at any ATM in the city.


Emergency numbers: 911  or  55 5658 1111

\subsection*{Language and Etiquette}
Spanish is the most spoken language in the city. You will only find English translations in a few places.  However, people are usually friendly and willing to help.

Some neighborhoods have a more international presence. For example, ``La Condesa'' and ``La Roma'' are neighborhoods with trendy restaurants and cafes where English is more widely spoken.

Tipping is not mandatory, but it is widely expected (around 10\% to 15\%).  It is  also customary to leave service tips for valets, gas station attendants, bellboys and similar service providers.

\subsection*{Weather}

June and July are the rainiest months of the year. Remember to check the weather forecast before you go out so you can be prepared.

Due to Mexico City's high altitude, sunscreen is recommended. The radiation tends to be intense.

\subsection*{Arrival}

You can take an official taxi from the booths in the airport hall (Yellow cab, Sitio 300, etc). They are usually safe and not overpriced, and you pay in advance.

You can take the Metrobus (red buses) that will take you to the city center. Take line 4 and get off at ``Bellas Artes'' or ``Hidalgo'' stop, depending on your hotel location. You will need to buy and top your MI transport card first

Uber is also a good option (but it is not always allowed at the airport, check first)

There is also a subway station (Metro terminal a\'erea) close to terminal 1, but it is not the best option to get into the city center (several transfers). This is not recommended if you carry heavy luggage or it is late at night.



