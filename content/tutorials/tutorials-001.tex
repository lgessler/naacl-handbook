%\begin{bio}
%
%\end{bio}
\clearpage
\begin{tutorial}
  {Catch Me If You GPT: Tutorial on Deepfake Texts}
  {tutorial-1}
  {\daydateyear, 09:00 -- 12:30}
  {Do\~na Adelita}

In recent years, Natural Language Generation (NLG) techniques have greatly advanced, especially in the realm of Large Language Models (LLMs). With respect to the quality of generated texts, it is no longer trivial to tell the difference between human-written and LLM-generated texts (i.e., deepfake texts). While this is a celebratory feat for NLG, it poses new security risks (e.g., the generation of misinformation). To combat this novel challenge, researchers have developed diverse techniques to detect deepfake texts. While this niche field of deepfake text detection is growing, the field of NLG is growing at a much faster rate, thus making it difficult to understand the complex interplay between state-of-the-art NLG methods and the detectability of their generated texts. To understand such inter-play, two new computational problems emerge: (1) Deepfake Text Attribution (DTA) and (2) Deepfake Text Obfuscation (DTO) problems, where the DTA problem is concerned with attributing the authorship of a given text to one of k NLG methods, while the DTO problem is to evade the authorship of a given text by modifying parts of the text. In this \textbf{cutting-edge tutorial}, therefore, we call attention to the serious security risk both emerging problems pose and give a comprehensive review of recent literature on the detection and obfuscation of deepfake text authorships. Our tutorial will be 3 hours long with a mix of lecture and hands-on examples for interactive audience participation.

\end{tutorial}
