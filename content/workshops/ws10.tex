\begin{wsschedulenolist}
{6th Workshop on NLP and Computational Social Science}
{10}
{ws10}
{ws10}
{Don Alberto 3}
{https://2024.naacl.org/program/workshops/}

Language is deeply intertwined with nearly all human social processes. We do not expect teenagers to speak like senior citizens, and we recognize the mutual dependency between language and the ways people interact with and even conceptualize the world. Although this interdependence is at the core of models in both natural language processing (NLP) and (computational) social sciences (CSS), these two fields are continuing to come together, with many opportunities for novel methods, research insights, and potential applications. Humans with different social attributes and cultural backgrounds (compared to bots and trolls) react to information spread online differently, and express their reactions using a large variety of language and content choices. Identifying and measuring bias based on language use in different online communities is another emerging area of research. Moreover, it has been shown that one can construct social variables from language and estimate the relationship between these social variables and measures in economics, politics, law, religion, anthropology and other fields.

This workshop aims to (1) advance the joint computational analysis of social sciences and language and (2) study how language can be used to measure social variables and their impact across disciplines, both by explicitly involving social scientists with NLP researchers, and other partners from both industry and academia.

This sixth edition of the NLP+CSS workshop builds on five successful years with hundreds of interdisciplinary submissions to make NLP techniques and insights standard practice in CSS research. Our focus is on NLP for social sciences: to continue the progress of CSS, and to integrate CSS with current trends and techniques in NLP.

\end{wsschedulenolist}
