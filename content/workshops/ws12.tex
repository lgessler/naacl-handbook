\begin{wsschedulenolist}
{The Fourth Workshop on Figurative Language Processing}
{12}
{ws12}
{ws12}
{Do\~na Genaro}
{http://sites.google.com/view/figlang2024}

Processing of figurative language is a rapidly growing area in NLP, including computational modeling of metaphors, idioms, puns, irony, sarcasm, simile, and other figures.  Characteristic to all areas of human activity (from poetic, ordinary, scientific, social media) and, thus, to all types of discourse, figurative language becomes an important problem for NLP systems. Its ubiquity in language has been established in a number of corpus studies and the role it plays in human reasoning has been confirmed in psychological experiments. This makes figurative language an important research area for computational and cognitive linguistics, and its automatic identification, interpretation and generation indispensable for any semantics-oriented NLP application.

The workshop will be the fourth edition of the biennial Workshop on Figurative Language Processing, whose first editions were held at NAACL 2018, ACL 2020 and EMNLP 2022, respectively. The workshop builds upon a long series of related workshops that the current organizers have been involved with: “Metaphor in NLP” series (2013-2016) and “Computational Approaches to Linguistic Creativity” series (2009-2010). We expand the scope to incorporate various types of figurative language, with the aim of maintaining and nourishing a community of NLP researchers interested in this topic. The main focus will be on computational modeling of figurative language, however papers on cognitive, linguistic, social, rhetorical, and applied aspects are also of interest, provided that they are presented within a computational, formal, or a quantitative framework.  Recent advancement in language models have led to several works on figurative language understanding (Chakrabarty et al 2022a; Chakrabarty et al 2022b; Liu et al 2022; Hu et al 2023) and generation (Stowe et al 2021; Chakrabarty et al 2021; Sun et al 2022; Tian et al 2021)  At the same time large language models have opened up opportunities to utilize figurative language in scientific (Kim et al 2023) as well as creative writing (Chakrabarty et al 2022c; Tian et al 2022). Additionally there have also been recent work on multimodal figurative language generation (Chakrabarty et al 2023; Akula et al 2023), understanding (Hessel et al 2023; Yosef et al 2023) and interpretation (Hwang et al 2023; Desai et al 2022; Kumar et al 2022).

\end{wsschedulenolist}
