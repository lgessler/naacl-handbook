\begin{wsschedulenolist}
{The Eleventh Workshop on NLP for Similar Languages, Varieties and Dialects (VarDial) 2024}
{6}
{ws6}
{ws6}
{Don Genaro}
{https://sites.google.com/view/vardial-2024}

VarDial is a well-established series of workshops promoting a forum for scholars working on a range of topics related to the study of diatopic language variation from a computational perspective.

The workshop deals with computational methods and language resources for closely related languages, language varieties, and dialects. We welcome papers dealing with one or more of the following topics:

\begin{itemize}
    \setlength{\itemsep}{-0.3ex}
    \item Corpora, resources, and tools for similar languages, varieties and dialects;
    \item Adaptation of tools (taggers, parsers) for similar languages, varieties and dialects;
    \item Evaluation of language resources and tools when applied to language varieties;
    \item Reusability of language resources in NLP applications (e.g., for machine translation, POS tagging, syntactic parsing, etc.);
    \item Corpus-driven studies in dialectology and language variation;
    \item Computational approaches to mutual intelligibility between dialects and similar languages;
    \item Automatic identification of lexical variation;
    \item Automatic classification of language varieties;
    \item Text similarity and adaptation between language varieties;
    \item Linguistic issues in the adaptation of language resources and tools (e.g., semantic discrepancies, lexical gaps, false friends);
    \item Machine translation between closely related languages, language varieties and dialects.
\end{itemize}

In addition to the topics listed above, we also welcome papers dealing with diachronic language variation (e.g. phylogenetic methods, historical dialects).

Papers presented at the past editions of VarDial focused on machine translation between closely related languages and language varieties, adaptation of POS taggers and parsers for similar languages and language varieties, compilation of corpora, spelling normalization, computational approaches to the study of mutual intelligibility, and the automatic identification of similar languages and dialects.



\end{wsschedulenolist}
