\begin{wsschedulenolist}
{Workshops on Data Science with Human in the Loop}
{7}
{ws7}
{ws7}
{Don Juli\'an}
{https://www.dashworkshops.org/}

Data science is fundamentally an iterative process requiring collaboration between humans and computers. Human insight and expertise are critical at all stages of the data science pipeline, starting with problem formulation and continuing with data collection, data cleaning and preparation, data and knowledge representation, data annotation and labeling, selection and evaluation of unsupervised and supervised models, knowledge discovery, and communicating the results. In order to unleash the full potential of data science, we need to improve our understanding about the best modalities of human and computer cooperation along the data science pipeline. This will allow us to design better interfaces and tools for cooperation between humans and computers in data science.

The goal of this workshop is to stimulate research on human-computer cooperation challenges in data science. We aim to bring together interdisciplinary researchers from academia, research labs and practice to share, exchange, learn, and develop preliminary results, new concepts, principle, and methodologies for understanding and improving human-computer cooperation along data science pipelines. We expect the workshop to help develop and grow a strong community of researchers who are interested in this topic, and yield future collaborations and scientific exchanges at the intersection of data mining, machine learning, natural language processing, computer vision, information retrieval, data and knowledge management, human factors, human-computer interaction, data visualization, and user interfaces. The workshop will focus on understanding how to leverage respective strengths of humans and computers along the data science pipeline and in a wide range of data science tasks and real-life applications.

\end{wsschedulenolist}
