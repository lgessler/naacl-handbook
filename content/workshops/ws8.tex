\begin{wsschedulenolist}
{21st SIGMORPHON Workshop on Computational Research in Phonetics, Phonology, and Morphology}
{8}
{ws8}
{ws8}
{Do\~na Socorro}
{https://sigmorphon.github.io/workshops/2024/}

SIGMORPHON aims to bring together researchers interested in applying computational techniques to problems in morphology, phonology, and phonetics. Work that addresses orthographic issues is also welcome. Papers will be on substantial, original, and unpublished research on these topics, potentially including strong work in progress. Appropriate topics include (but are not limited to) the following as they relate to the areas of the workshop:

\begin{enumerate}
\item New formalisms, computational treatments, or probabilistic models of existing linguistic formalisms
\item Unsupervised, semi-supervised, or machine learning of linguistic knowledge
\item Analysis or exploitation of multilingual, multi-dialectal, or diachronic data
\item Integration of morphology, phonology, or phonetics with other NLP tasks
\item Algorithms for string analysis and manipulation, including finite-state methods
\item Models of psycholinguistic experiments
\item Approaches to orthographic variation
\item Approaches to morphological reinflection
\item Corpus linguistics
\item Machine transliteration and back-transliteration
\item Morpheme identification and word segmentation
\item Speech technologies relating to phonetics or phonology
\item Speech science (both production and comprehension)
\item Instructional technologies for second-language learners
\item Tools and resources
\end{enumerate}

SIGMORPHON encourages interaction between work in computational linguistics and work in theoretical phonetics, phonology and morphology, and to ensure that each of these fields profits from the interaction. Our recent meetings have been successful in this regard, and we hope to see this continue in 2024.

Many mainstream linguists studying phonetics, phonology and morphology are employing computational tools and models that are of considerable interest to computational linguists. Similarly, models and tools developed by and for computational linguists may be of interest to theoretical linguists working in these areas. This workshop provides a forum for these researchers to interact and become exposed to each others’ ideas and research.

\end{wsschedulenolist}
