\begin{wsschedulenolist}
{The 6th Clinical Natural Language Processing Workshop}
{9}
{ws9}
{ws9}
{Don Alberto 2}
{https://clinical-nlp.github.io/2024/committee.html}

Clinical text is growing rapidly as electronic health records become pervasive. Much of the information recorded in a clinical encounter is located exclusively in provider narrative notes, which makes them indispensable for supplementing structured clinical data in order to better understand patient state and care provided. The methods and tools developed for the clinical domain have historically lagged behind the scientific advances in the general-domain NLP. Despite the substantial recent strides in clinical NLP, a substantial gap remains. The goal of this workshop is to address this gap by establishing a regular event in CL conferences that brings together researchers interested in developing state-of-the-art methods for the clinical domain. The focus is on improving NLP technology to enable clinical applications, and specifically, information extraction and modeling of narrative provider notes from electronic health records, patient encounter transcripts, and other clinical narratives.

Relevant topics for the workshop include, but are not limited to:

\begin{enumerate}
\item Modeling clinical text in standard NLP tasks (tagging, chunking, parsing, entity identification, entity linking/normalization, relation extraction, coreference, summarization, etc.)
\item De-identification and other handling of protected health information
\item Disease detection and other coding of clinical documents (e.g., ICD)
\item Structure of clinical documents (e.g., section identification)
\item Information extraction from clinical text
\item Integration of structured and textual data for clinical tasks
\item Domain adaptation and transfer learning techniques for clinical data
\item Generation of clinical notes: summarization, image-to-text, generation of notes from clinical conversations, etc.
\item Annotation schemes and annotation methodology for clinical data
\item Evaluation techniques for the clinical domain
\item Bias and fairness in clinical text
\end{enumerate}

In 2024, Clinical NLP will encourage submissions from the following special tracks:
\begin{enumerate}
\item Clinical NLP in low-resource settings (e.g., languages other than English)
\item Clinical NLP for clinical conversations (e.g., doctor-patient)
\item Risk analysis of large language models for clinical NLP (e.g., privacy, bias)
\end{enumerate}


\end{wsschedulenolist}
